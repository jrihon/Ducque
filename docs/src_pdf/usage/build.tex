\subsection{Build}

\begin{verbatim}
$ Ducque --build INPUTFILE
      The inputfile is read in and the sequence is built accordingly.
      ' * ' : mandatory
      ' + ' : additional in the GUI

INPUTFILE : [
--sequence SEQUENCE *
  Only valid input in the file just a string of nucleotides (comma-delimited).
      Example: --sequence   dT, dC, dA, dA, dC, dG, dG, dT, dA

--complement COMPLEMENT *
  The complement flag denotes the sequence of the complementary strand.
  A list of nucleotides is also a valid input (comma-delimited).
      Example: --complement homo
      Example: --complement rA, rG, rT, rT, rG, rC, rC, rA, rT

--pdbname PDBNAME
  To prompt the name of the pdbfile for the outputted structure.
  If none is given, this defaults to the name of the INPUTFILE.
        Example: --pdb foobar.pdb
]
GUI : [
--filename FILENAME + 
  The name of the INPUTFILE to write to.
]
\end{verbatim}
%
The \textbf{Build} module allows to build a structure with a simple query. The \emph{sequence} and \emph{complement} flags are mandated. The \emph{pdbname} allows you to name the produced pdb structure. Defaults to the name of the prompted INPUTFILE. \\ 
The GUI module has an added flag \emph{filename}, to which you need to prompt a filename, which is the INPUTFILE that is used to build the pdb structure. Defaults to \textit{random\_sequence}.
\pagebreak
