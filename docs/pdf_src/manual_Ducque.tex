\documentclass[a4paper, 11pt]{article}      % Documentclass article
\usepackage[utf8]{inputenc}                 % Type of encoding of the letters
\usepackage[T1]{fontenc}
\usepackage{fullpage}                       % The margins of the page are smaller
\usepackage{titling}                        % Flush left the title and the authors
\usepackage{authblk}                        % Used for the author affiliations.
\usepackage{natbib}                         % Used for styling citations
\usepackage{graphicx}                       % Used to import graphic documents, in pdf and png mostly
\usepackage{amsmath}                        % Mathematical package
\usepackage{bm}                             % Use to make things in bold
\usepackage{siunitx}                        % Used to converts floats as SI units
\usepackage{wrapfig}                        % Use this to wrap text around a figure
\usepackage[superscript]{cite}              % Used to superscript the numbers for the citations in the text
\usepackage[super]{nth}                     % To superscript when counting up numbers
\usepackage{float}                          % float package for the [H] option to the figure
\usepackage{hyperref}                       % Make hyperlink references
\usepackage{lmodern}                        % To be able to use any fontsize
\usepackage{verbatim}                       % verbatim coding examples
%\usepackage[most]{tcolorbox}
\usepackage{booktabs}                       % For controlling the width of \hrule
\usepackage{colortbl}                       % to colour in a cell in a table
\usepackage{multirow}                       % same as multicolumn
\usepackage{lipsum}                         % lorem ipsum filler text
\usepackage{epigraph}                       % display quotes
\usepackage{tipa}                           % IPA phonetic language characters
\usepackage{enumitem}                       % to make list envs not indented
%\usepackage[usenames,dvipsnames]{color}     % create coloured backgrounds in text
\usepackage{minted}                         % syntax highlighting for code in latex; https://www.overleaf.com/learn/latex/Code_Highlighting_with_minted


% Make the paper in sans-serif font
\renewcommand{\familydefault}{\sfdefault}
% Set up the way you show a url or href command
\hypersetup{colorlinks=true, linkcolor=blue, filecolor=magenta, urlcolor=cyan}
% set up pygmentize style
\usemintedstyle{murphy}

% TITLE
\title{\textbf{\Huge The manual to Ducque}}

% LIST OF AUTHORS
\author[1,*]{Jérôme Rihon}

% AFFILIATIONS
\affil[1]{\small KU Leuven, Rega Institute for Medical Research, Medicinal Chemistry, Herestraat 49 - Box 1041, 3000 Leuven, Belgium}
\affil[*]{\small \textit{Corresponding author, maintainer}}

% makes the date{} empty and does not take up extra whitespace
\date{}
\predate{} 
\postdate{}

% define colors
\definecolor{gray1}{gray}{0.8}
% ------------------------------------
%               START
% ------------------------------------
\begin{document}

\maketitle

\footnote[69]{
        \textit{
All rights reserved to the Laboratory of Medicinal Chemistry Rega Institute of Medical Research Herestraat 49, 3000 Leuven, Belgium. Katholieke Universiteit Leuven (KUL).
                }
}


% ---------------
%   ToC
% ---------------
\pagebreak
\tableofcontents

% ---------------
%   Voorblad
% ---------------
% TODO Make a front cover for the manual

% ---------------
%   Contents
% ---------------
\pagebreak

\section{Introduction}

\epigraph{There are no mistakes, only happy little accidents.}{Bob Ross}

\begin{center}
    \textbf{A software for the purpose of building native and synthetic nucleic acid duplexes.}
\end{center}

\noindent Ducque \colorbox{gray!20}{(IPA : /d\textipa{2}k/)} stands for :
\begin{itemize}[leftmargin=*]
    \setlength{\itemsep}{-1mm}
    \item[\textbf{D}] Acronyms
    \item[\textbf{U}] Are
    \item[\textbf{C}] Rather
    \item[\textbf{Q}]
    \item[\textbf{U}]
    \item[\textbf{E}] Tedious
\end{itemize}
%
%
%
The name was inspired much in the same way that the ORCA Quantum Mechanics package\cite{Neese2020Orca, Neese2022ORCA} was named.
At the time of development, I had different name for Ducque. During one of my evaluation moments, one of the members of my jury pointed out that with that name, I would probably not get enough traction, since Google searches gaves millions of results.
Given that fact, I tried coming up with a new name for the model builder.
Weeks later, as I was working, my hat that was resting on my desk in front of me caught my eye.
You see, my hat has a tiny duck embroided on the front. With this idea, I "researched" ways to rewrite the word and get as little search engine results as possible, without degenerating the word itself and keeping it phonetically somewhat correct. Thusly, Ducque was born (a second time).


\subsection{Modules}
Ducque has five modules the user can access.
\begin{enumerate}
    \item The \textbf{build} module, which is the primary module for to use. This will build us the duplex structures
we want to use for further
\item The \textbf{transmute} will be used, together with an elaborate input, to convert the ‘pdb‘ file to a
suitable input for Ducque to use as a building block. The appropriate format has been settled to
be the very simple ‘json‘ format.
\item For testing purposes or just to be able to generate randomised sequences, there is a \textbf{randomise}
module, which generates a randomised duplex structure based on a set of given inputs.
    \item To be able to convert ‘xyz‘ formatted molecule structures to a ‘pdb‘ format, I have included a
        \textbf{xyz\_pdb} module. Since I am an avid ORCA user and the ‘xyz‘ format is often outputted, I
wanted to make it more managable to format a ‘pdb‘ file for myself during development.
\item To have clickable objects for the standard non-terminal user, there is a simple \textbf{GUI} that can be
used to employ the following modules : build, transmute and randomise. The reason that the
xyz\_pdb is not included is because I did not find a nice way to design a simple gui. Secondly,
other QM programs might have different outputs of molecule structures, the ‘xyz‘ format can be
rather niche and therefor not worth to effort to design a good gui for. Chimera and PyMol can
definitely convert from one format to the other if one cannot work through the CLI.
    
\end{enumerate}
\subsection{Package structure}
\textbf{path/to/Ducque}
%
\begin{itemize}[leftmargin=*]
    \setlength{\itemsep}{-1mm}
    \item[$\rightarrow$] \textbf{bin/:}
    \item[$\rightarrow$] \textbf{docs/:}
    \item[$\rightarrow$] \textbf{json/:}
    \item[$\rightarrow$] \textbf{puckerdata/:}
    \item[$\rightarrow$] \textbf{src/:}
    \item[$\rightarrow$] \textbf{tests/:}
    \item[$\rightarrow$] \textbf{transmute/:}
    \item[$\rightarrow$] \textbf{xyz/:}
    \item[$\rightarrow$] LICENSE:
    \item[$\rightarrow$] README.md :
    \item[$\rightarrow$] setup.sh :
\end{itemize}


\pagebreak
\section{Usage}
\subsection{Build}

\small
\begin{verbatim}
$ Ducque --build INPUTFILE

The inputfile is read in and the sequence is built accordingly.
At any given time, there are two (2) flags in total that should be involved in the Ducque.

--sequence SEQUENCE
    Only valid input in the file just a string of the required nucleotides (comma-delimited).
        Example: --sequence dT, dC, dA, dA, dC, dG, dG, dT, dA

--complement COMPLEMENT
    The complement flag denotes the structure of the complementary strand
    A list of nucleotides is also a valid input, if one wants a specific complementary strand. 
        Example: --complement homo
        Example: --complement dT, dA, dC, dC, dG, dT, dT, dG, dA
    
\end{verbatim}
\normalsize

\pagebreak

\subsection{Transmute}

\lipsum[1-2]


\pagebreak

\subsection{Randomise}

\lipsum[1-2]


\pagebreak
\subsection{XYZ to PDB}

\small
\begin{verbatim}
$ Ducque --xyz_pdb INPUTFILE
The inputfile is read, the xyz file used as an input to output a well formatted pdb.

There are three (3) flags involved in converting a `xyz` coordinate file to a `pdb` file.
--xyz XYZ
    The name of the file of the molecule you want to convert to pdb
       Example --xyz dna_2endo.xyz

--residue RESIDUE
    The identifier of the nucleo(s)(t)ide.
    See the PDB format %link-to-pdb-format.
    NB : Ducque does not allow custom nucleic acid chemistries with a RESIDUE unequal to three!
        (in this example, dXA is equal to deoxy Xylose nucleic acid with an adenine base)
        Example: --atomID dXY

--atomname_list ATOMNAME_LIST
    The ordered list of atoms that belong in the 'Atom name' column in a pdb file.
    See the PDB format %link-to-pdb-format.
    The order needs to so that it follows the order of the atoms from the xyz file. 
    DISCLAIMER : the responsability is with the end-user to see everything is correct.
        Example: --atomname_list O5', C5', H5'1, H5'2, C4' ··· , O3' 
\end{verbatim}
\normalsize

\pagebreak
\subsection{Graphic User Interface (GUI)}


\pagebreak


\subsection{User implementation of custom nucleic acids}
\textbf{Before adding a new chemistry}\\
In \$TRANSMUTETOOLS :
\begin{enumerate}
    \item Add to the \$NUCLEOSIDEDICT the type of chemistry it is. The key should be in all caps.
    \item Add to the \$LINKERDICT the type of linker it corresponds with. The key should be in all caps.
    \item This only serves the purpose of identifying the json file when opening it as the user. This information is not used in the generation of nucleic acid duplexes.
    
\end{enumerate}
\textbf{Before generating a duplex with the new chemistry}\\
In \$REPOSITORY :
\begin{enumerate}
    \item Add to the \$CHEMCODEX the most stable conformation of the chemistry you're adding to the library.
    \item Add to the \$CONFCODEX all the conformations that you have at your disposal of the chemistry you're adding to the library. The key in both these abbreviated name of the nucleic acid chemistry.
    \item Add to the \$BACKBONECODEX the sugar linker backbone of the chemistry you're adding to the library.
\end{enumerate}



\pagebreak
\section{Installation}
\subsection{Set-up}
To access the Ducque software from anywhere on your machine, add the following line to your \textbf{\$HOME/.bashrc}.
Where path/to/program is the path to where you've installed Ducque. \$pwd inside the Ducque directory if you're unsure.
\begin{verbatim}
export PATH=\$PATH:path/to/program/Ducque/bin
\end{verbatim}

\subsection{Environment}
To run this project, you will need to add the libraries to your PythonEnv : `NumPy`, `SciPy`\\
Installation (conda or pip)\\
`\$ pip install numpy` | `\$ conda install -c numpy `\\

`\$ pip install scipy` | `\$ conda install -c scipy `\\
\\
Ducque also employs `sys`, `os`, `tkinter` and `json`. These are built-in libraries, so no need to install these additionally.\\
Depending on your `python3` version, you might have to install `tkinter` separately.\\
\\
Check if you have tkinter installed : `\$ python3 -m tkinter`
```




% ---------------
%   BIBLIOGRAPHY
% ---------------
\pagebreak
% add bibliography as a section to the Table Of Contents and name it `References`
\addcontentsline{toc}{section}{References}
\renewcommand{\bibfont}{\small}
\renewcommand{\refname}{\textbf{References}}
\bibliographystyle{unsrt}
\bibliography{references/references.bib}


        
% ------------------------------------
%               END
% ------------------------------------
\end{document}
