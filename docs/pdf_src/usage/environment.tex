
\pagebreak
\section{Installation}
\subsection{Set-up}
To access the Ducque software from anywhere on your machine, add the following line to your \textbf{\$HOME/.bashrc}.
Where path/to/program is the path to where you've installed Ducque. \$pwd inside the Ducque directory if you're unsure.
\begin{verbatim}
export PATH=\$PATH:path/to/program/Ducque/bin
\end{verbatim}

\begin{verbatim}
$ cd path/to/Ducque   # Go to the Ducque directory, where you installed it
$ chmod +x setup.sh   # Give permission to run as an executable script
$ ./setup.sh          # Run the setup.sh script
\end{verbatim}

\subsection{Environment}
To run this project, you will need to add the libraries to your PythonEnv : `NumPy`, `SciPy`\\
Installation (conda or pip)\\
`\$ pip install numpy` | `\$ conda install -c numpy `\\

`\$ pip install scipy` | `\$ conda install -c scipy `\\
\\
Ducque also employs `sys`, `os`, `tkinter` and `json`. These are built-in libraries, so no need to install these additionally.\\
Depending on your `python3` version, you might have to install `tkinter` separately.\\
\\
Check if you have tkinter installed : `\$ python3 -m tkinter`
```
