
\pagebreak
\section{Installation}
\subsection{Set-up}
To access the Ducque software from anywhere on your machine, add the following line to your \verb $HOME/.bashrc  .
Where path/to/program is the path to where you've installed Ducque.\\
Use the \verb $pwd  command inside the Ducque directory if you're unsure.
Depending on your \verb python3  version, you might have to install tkinter separately.
\begin{minted}
[
frame=lines,
]
{bash}
export PATH=$PATH:path/to/program/Ducque/bin # in the $HOME/.bashrc
\end{minted}

\begin{minted}
[
frame=lines,
]
{bash}
$ cd path/to/Ducque   # Go to the Ducque directory, where you installed it
$ chmod +x setup.sh   # Give permission to run as an executable script
$ ./setup.sh          # Run the setup.sh script
    
\end{minted}

\subsection{Environment}
%To run this project, you will need to add the libraries to your PythonEnv : `NumPy`, `SciPy`\\
%Installation (conda or pip)\\
\begin{minted}
[
frame=lines,
]
{bash}
# Using the pip package manager
$ pip install numpy 
$ pip install scipy

# Using the conda package manager
$ conda install -c numpy
$ conda install -c scipy

# Check if you have tkinter installed :
 $ python3 -m tkinter # if not succesful ... 
 $ sudo apt-get install python3-tk # install this
\end{minted}
