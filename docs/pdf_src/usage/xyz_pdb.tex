\pagebreak
\subsection{XYZ to PDB}

\small
\begin{verbatim}
$ Ducque --xyz_pdb INPUTFILE
The inputfile is read, the xyz file used as an input to output a well formatted pdb.

There are three (3) flags involved in converting a `xyz` coordinate file to a `pdb` file.
--xyz XYZ
    The name of the file of the molecule you want to convert to pdb
       Example --xyz dna_2endo.xyz

--residue RESIDUE
    The identifier of the nucleo(s)(t)ide.
    See the PDB format %link-to-pdb-format.
    NB : Ducque does not allow custom nucleic acid chemistries with a RESIDUE unequal to three!
        (in this example, dXA is equal to deoxy Xylose nucleic acid with an adenine base)
        Example: --atomID dXY

--atomname_list ATOMNAME_LIST
    The ordered list of atoms that belong in the 'Atom name' column in a pdb file.
    See the PDB format %link-to-pdb-format.
    The order needs to so that it follows the order of the atoms from the xyz file. 
    DISCLAIMER : the responsability is with the end-user to see everything is correct.
        Example: --atomname_list O5', C5', H5'1, H5'2, C4' ··· , O3' 
\end{verbatim}
\normalsize
