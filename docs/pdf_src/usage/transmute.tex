\pagebreak
\subsection{Transmute}


\small
\begin{verbatim}
$ Ducque --transmute INPUTFILE

The inputfile is read in and the json file is formatted accordingly.
There are five (5) flags needed to convert from `pdb` to a `json` file format.

--pdb PDB
    The name of the file of the structure you want to convert to json
        Example: --pdb dna_A.pdb

--chemistry ID
    The chemistry that defines the given nucleo(s)(t)ide
        Example: --chemistry DNA

--comformation CONFORMATION
    The conformation that denotes the nucleic acid.
    Used to name the output .json file.
    Allows multiple conformers for a given chemistry.
    Used when building the complementary strand.
    The complementary strand is then fitted onto the leading strand.
        Example: --conformation 2endo
        Example: --conformation 1-4boat

--moiety MOIETY
    The moiety that the structure defines. Should either be "nucleoside" or "linker"
        Example: --moiety nucleoside

--bondangles ALPHA, BETA, GAMMA, DELTA, EPSILON, ZETA, CHI
    The bond angles involved in the backbone and the anomeric carbon, DEGREES
    Comma-delimited string.
        Example: --bondangles 101.407, 118.980, 110.017, 115.788, 111.943, 119.045, 126.013

--dihedrals ALPHA, BETA, GAMMA, DELTA, EPSILON, ZETA, CHI
    The dihedrals involved in the backbone and the anomeric carbon, DEGREES
    Comma-delimited string.
        Example: --dihedrals -39.246, -151.431, 30.929, 156.517, 159.171, -98.922, -99.315
\end{verbatim}
\normalsize

\noindent Standard JSON structure of the Ducque input
\small
\begin{verbatim}

    { 
	pdb_properties : { Coordinates : [X, Y, Z] ,
                       Shape : (rows, columns) ,
                       Atoms : [O5', C5', H5'1, H5'2, C4' ··· , O3'] ,
                       Symbol : [O, C, H, H, C  ··· , O ] 
                      }

	identity: [ Chemistry, Abbreviation, Residue name, Base ]

	angles : { dihedrals : { Alpha : X,
                             Beta : X,
                             Gamma : X,
                             Delta : X,
                             Epsilon : X,
                             Zeta : X,
                             Chi : X
                             }
             }

             { bond_angles : { Alpha : X, 
                               Beta : X,
                               Gamma : X,
                               Delta : X,
                               Epsilon : X,
                               Chi : X
                               }
             }
    }
    
\end{verbatim}
\normalsize
