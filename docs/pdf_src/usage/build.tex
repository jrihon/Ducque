\pagebreak
\section{Usage}
\subsection{Build}

\small
\begin{verbatim}
$ Ducque --build INPUTFILE

The inputfile is read in and the sequence is built accordingly.
At any given time, there are two (2) flags in total that should be involved in the Ducque.

--sequence SEQUENCE
    Only valid input in the file just a string of the required nucleotides (comma-delimited).
        Example: --sequence dT, dC, dA, dA, dC, dG, dG, dT, dA

--complement COMPLEMENT
    The complement flag denotes the structure of the complementary strand
    A list of nucleotides is also a valid input, if one wants a specific complementary strand. 
        Example: --complement homo
        Example: --complement dT, dA, dC, dC, dG, dT, dT, dG, dA
    
--pdbname PDBNAME
    To prompt the name of the pdbfile for the outputted structure.
    If none is given, this defaults to the name of the queried inputfile.
\end{verbatim}
\normalsize
